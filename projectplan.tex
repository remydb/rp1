\documentclass{article}
\begin{document}

\title{Monitoring fiber optical cables over 1625nm\\Project plan}
\author{Stefan Plug\\Remy de Boer}
\date{\today}
\maketitle

\section{Research question}
For this research project we will look at the possibilities of monitoring fibreglass cables by means of a separate wavelength used exclusively for monitoring.
The research question will be as follows:
\begin{quote}
\textit{
Is it possible to monitor vital statistics of optical fibreglass cables using the 1625nm channel
}
\end{quote}

The words 'vital statistics' in the main research question has been kept deliberately vague because there are a multitude of statistics which could be monitored.\\
When monitoring fibreglass cables, a multitude of values are measured.
These values include the signal strength, bit error rate, attenuation, dispersion and a myriad of other values.
During our research we will decide which of these values we can effectively monitor using the relatively basic 1625nm optical modules.

The ITU-T has created a set of standards which show what the wavelength coverage each type of single-mode cable type is designed for \cite[p.~21]{refguide:2011} 
Although we will not test every cable type for their specific 1625nm attenuation properties, we should keep in mind that attenuation of the 1625nm channel monitoring signal might be higher than that of the main data channels.





\section{Planning}

\section{Bibliography}
\bibliographystyle{plain}
\bibliography{bibliography}

\end{document}
