\documentclass{article}

\begin{document}

\title{Develop/prototype a new optical network monitoring tool. }
\author{Stefan Plug, Remy de Boer}
\date{\today}
\maketitle

\begin{tabular}{|c|c|c|}
\hline 
Version number & Date & Comment \\ 
\hline 
0.1 & 03-01-2013 & Start of of document \\ 
\hline 
\end{tabular} 

\tableofcontents

\section{preface}

\section{Summary}

\section{Research question}

\section{Actual research chapters}

\section{1625nm cabable cables}
This ITU-T standard shows us that the 1625nm channel which resides at the beginning of the L-band is not officially supported by several of the singel-mode optical cable standards as show in table \ref{tab:singel-mode_types}.

\begin{table}[h]
\centering
\label{tab:singel-mode_types}
\caption{ITU-T Singel-mode cable standards}
\begin{tabular}{|c|c|c|}
\hline 
ITU-T code & Wavelength coverage\\ 
\hline 
G.652.A & O and C bands \\ 
\hline
G.652.B & O and C+L bands \\
\hline
G.652.C & From O to C bands \\
\hline
G.652.D & From O to L bands \\
\hline
\end{tabular} 
\end{table}

The cables which do not officially support the 1625nm channel are expected to generate a higher attenuation effect. Although we will not try to test this effect for all the different kinds of fiber, we should keep this in mind during testing.

\section{Solution proof of concept}

\section{Conclusion}

\section{Bibliography}

\end{document}
